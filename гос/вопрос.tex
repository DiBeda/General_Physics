\documentclass[a4paper, 12pt]{article}
\usepackage[T2A,T1]{fontenc}
\usepackage[utf8]{inputenc}
\usepackage[english, russian]{babel}
\usepackage{graphicx}
\graphicspath{{img/}}
\usepackage[hcentering, bindingoffset = 10mm, right = 15 mm, left = 15 mm, top=20mm, bottom = 20 mm]{geometry}
\usepackage{amsmath, amstext}
\usepackage{float}
\usepackage{wasysym}
\setlength{\parindent}{0em}
\setlength{\parskip}{0.5em}
\renewcommand{\arraystretch}{1.2}

\newenvironment{bottompar}{\par\vspace*{\fill}}{\clearpage}
 
\begin{document}

\begin{titlepage}
\newcommand{\HRule}{\rule{\linewidth}{0.5mm}} % Defines a new command for the horizontal lines, change thickness here

\center % Center everything on the page
 
\textsc{\LARGE Московский\\[-0.2cm]Физико-Технический Институт\\[0.1cm]\large (государственный университет)}\\[1.5cm] % Name of your university/college
\textsc{\Large Кафедра общей физики}\\[0.1cm] % Major heading such as course name
\textsc{\large Вопрос по выбору, ГКЭ}\\[0.5cm] % Minor heading such as course title

\HRule
\\[0.8cm]
{ \huge \bfseries Фазовые переходы второго рода\\[0.3cm]  и их механическая аналогия}
\\[0.8cm] % Title of your document
\HRule
\\[1.5cm]

\begin{flushleft} \large
	\textsf{Студент}\\[0.1cm]
	Ушаков Роман \\
	511 группа
\end{flushleft}

\begin{bottompar}
	\begin{center}
		\includegraphics[width = 80 mm]{logo.jpg}
	\end{center}
	{\large \today}

\end{bottompar}
\vfill
\end{titlepage}


\subsection*{Теория}

\emph{Фаза}~--- макроскопическая физически однородная часть вещества, отделённая от остальных частей границами раздела, так что она может быть извлечена из системы механическим путём.

Рассмотрим систему, которая состоит из двух фаз 1 и 2, которые могут превращаться друг в друга. Для механического и теплового равновесия необходимо, чтобы $P_1 = P_2$ и $T_1 = T_2$. Пусть $m_1$ и $m_2$~--- соответственно массы фаз, а $\varphi_1$ и $\varphi_2$  ~--- удельные термодинамические потенциалы вещества в этих фазах. Тогда очевидно, что термодинамический потенциал всей системы $\Phi = m_1 \varphi_1 + m_2 \varphi_2$. Напомним, что термодинамический потенциал системы по определению равен $\Phi = \Phi(P, T) = U - TS + PV$, поэтому если при фазовых превращениях $T = ~const$ и $P =~const$, то $\varphi_1$ и $\varphi_2$ не будут меняться. Следовательно, при фазовых превращениях меняются только массы $m_1$ и $m_2$, и эти превращения должны происходить в таком направлении, чтобы $\Phi$ принял наименьшее значение, которое возможно в рассматриваемых условиях. Поэтому только при условии $\varphi_1(P, T) = \varphi_2(P, T)$ фазы будут находиться в равновесии друг с другом.

Мы получили следующее важное следствие: при всех изменениях состояния вещества его удельный термодинамический потенциал всегда меняется непрерывно (в отличии от других величин ~--- удельного объема, удельной энтропии и пр.).

Фазовые переходы, при которых первые производные функции $\varphi(P, T)$ меняются скачкообразно, называются \emph{фазовыми превращениями первого рода}. Фазовые превращения, в которых первые производные остаются непрерывными, а вторые производные меняются скачкообразно, называются \emph{фазовыми переходами второго рода}.

Так как $ d \varphi = - s dT + v d P$, то
$$
	s = - \left( \frac{\partial \varphi }{\partial T} \right)_{P} \text{ и } 
	v = \left( \frac{\partial \varphi }{\partial P} \right)_{T} .
$$

Следовательно, фазовые переходы первого рода характеризуются скачкообразным изменением либо удельной энтропии $s$ или удельного объема $v$, либо обеих величин сразу. Скачкообразное изменение энтропии означает выделение или поглощение тепла. Отсюда следует, что фазовые переходы второго рода не сопровождаются выделением или поглощением теплоты, а также изменением удельного объема вещества. Таким образом, фазовый переход второго рода является непрерывным в том смысле, что состояние тела меняется непрерывным образом.

Фазовые переходы второго рода сопровождаются изменением симметрии вещества. Изменение симметрии может быть связано со смещением атомов определённого типа в кристаллической решётке, либо с изменением упорядоченности вещества.

В общую теорию фазовых переходов второго рода входит некоторый \emph{параметр порядка} $\eta$, отличный от нуля в упорядоченной фазе и равный нулю в неупорядоченной. Например, в магнетизме параметр порядка характеризует выстраивание магнитных моментов атомов вдоль направления макроскопической намагниченности. Обычно переход из упорядоченного состояния в неупорядоченное происходит при увеличении температуры (однако, есть и исключения). 

Пусть теперь термодинамический потенциал системы есть функция от давления, температуры и еще параметра порядка: $\Phi(P, T, \eta)$. Заметим, что переменная $\eta$  ``неравноправна'' с переменными $P$ и $T$, потому что давление и температура могут быть заданы произвольно, а реально осуществленное значение порядка должно быть определено из условия минимальности $\Phi$ при заданных $P$ и $T$.

Непрерывность изменения состояния при фазовом переходе второго рода математически выражается в том, что вблизи точки перехода величина $\eta$ может принимать сколь угодно малые значения. Поэтому при заданном давлении $P$ и температуре $T$ можно разложить $\Phi$ по степеням $\eta$:
$$ 
	\Phi(P, T, \eta) = \Phi_0 + a_1 \eta + a_2 \eta^2 + a_3 \eta^3 + a_4 \eta^4,
$$
где $a_i = a_i(P, T)$.

Можно показать*, что в отсутствии внешних полей термодинамические потенциалы являются четными функциями:

$$
	\Phi(P, T, \eta) = \Phi_0 + \frac{1}{2} A(P, T) \cdot \eta^2 + \frac{1}{4} B(P, T) \cdot \eta^4
$$

Равновесные значения $\eta$ находим из условия минимума функции $\Phi$:

$$
\begin{cases}
	\eta = 0 \\
	\eta = \pm \sqrt{\frac{-A}{B}}
\end{cases}
$$

B классической теории фазовых переходов предполагается, что коэффициенты $A$ и $B$ зависят только от давления и температуры, причем 

$$
A(P, T) = a(P, T_{cr}) (T - T_{cr}) 
$$
$$
B(P, T) = b(P, T_{cr}) \text{, } b(P, T_{cr}) > 0.
$$

Зависимость $\eta$ от температуры вблизи точки перехода в несимметричной фазе определяется из условия минимальности $\Phi$ как функции от $\eta$. Приравнивая к нулю производную $\frac{\partial \Phi}{\partial \eta}$ находим: 
$$
	\eta = \pm \sqrt{ \frac{a \cdot (T - T_{cr})}{B} } \text{, }
$$

Итак, мы получили важное следствие: $\eta^2 \sim (T - T_{cr})$. 

Можно рассмотреть важный частный пример: поведение ферромагнетиков вблизи точки Кюри. Рассматривая зависимость термодинамического потенциала как функцию намагниченности и внешнего поля $\Phi = \Phi(\vec{M}, \vec{H})$ можно получить аналогичный результат для спонтанной намагниченности в ферромагнитной фазе:

$$
	M^2 \sim (T - T_{cr})
$$


\subsection*{Механическая аналогия}

Рассмотрим следующий механический эксперимент. 

\subsubsection*{Оборудование}

\begin{enumerate}
	\item Динамик с приклеенной к мембране мензуркой
	\item 50-100 шариков кускуса
	\item Генератор импульсов Г3-33
	\item Мобильный телефон iPhone 6
	\item Миллиметровка
	\item Карандаш, ручка
	\item Линейка
	\item Изолента
	\item Картон
\end{enumerate}

\begin{figure}[h]
	\begin{center}
		\begin{minipage}{0.45 \linewidth}
			\includegraphics[height=8cm]{img/scheme.jpg}
		\end{minipage}
		\qquad
		\begin{minipage}{0.45 \linewidth}
			\includegraphics[height=7cm]{img/zoom_scheme.jpg}
		\end{minipage}
	\end{center}
	\caption{Cхема установки}
\end{figure}

Теперь перейдем непосредственно к аналогии. В нашем случае возбуждению атомов соответствует кинетической энергии частиц, приводимых в движение мембраной динамика. Эта энергия пропорциональна квадрату скорости колебаний динамика: $v^2 = f^2 \cdot A^2$, где $f$~--- частота колебаний, $A$~--- амплитуда колебаний мембраны динамика. Макроскопическое изменение, соответствующее фазовому переходу, ~--- собиранию шариков в одной из половин цилиндра, разделенных небольшой стенкой (аналогия с упорядочиванием магнитных моментов атомов). Коэффициент  $\left\lvert \frac{N_1 - N_2}{N_1 + N_2} \right\rvert$ хорошо подходит для параметра порядка.


\emph{Цель работы}: найти зависимость параметра порядка $\eta = \left\lvert \frac{N_1 - N_2}{N_1 + N_2} \right\rvert$ от $\left\vert A^2 - A_{cr}^2 \right\vert$.


Сначала проведем предварительный эксперимент. Насыпем кускус в обе половины равномерно, затем поставим максимальную амплитуду колебаний динамика. При таких колебаниях семена будут распределены равномерно (это соответствует неупорядоченным магнитным моментам). Теория предсказывает, что существует такая амплитуда (аналог точки Кюри), при которой семена перейдут в одну из половин из-за небольших неидеальностей в экспериментальной установке. Этот эффект можно пронаблюдать, плавно уменьшая амплитуду колебаний. Характерное время перемещения всех частиц (порядка 50) в одну из половин при критической амплитуде составляет порядка 20~30 минут.


\subsubsection*{Ход работы}

\begin{enumerate}
	\item Определение критической амплитуды фазового перехода
	\item Калибровка установки
	\item Определение показателя степени
\end{enumerate}

\emph{Определение критической амплитуды}

Подсоединим динамик к генератору с соблюдением полярности, затем насыпем 50 шариков кускуса в одну из половин цилиндра. Постепенно увеличивая амплитуду, будем записывать выходное напряжение генератора и количество шариков $N_1$ и $N_2$ в каждой половине сосуда.

\begin{figure}
	\includegraphics[scale=0.8]{img/N(V).jpg}
	\caption{зависимость $N_1$ и $N_2$ от амплитуды}
\end{figure}

\emph{Калибровка установки}

С помощью замедленной съемки на камере мобильного телефона iPhone 6 и миллиметровой бумаги найдём зависимость реальной амплитуды колебаний мембраны динамика от подаваемого вольтажа.

Есть и другой способ. С помощью изоленты можно прикрепить карандаш к верху пробирки и подносить миллеметровку к колеблющемуся карандашу. Такой способ определения колебаний также оказался достаточно точным.


\begin{figure}
	\includegraphics[scale=0.8]{img/A(V).jpg}
	\caption{зависимость амплитуды мембраны от напряжения}
\end{figure}

\emph{Определение показателя степени}

Построим график зависимости $\eta$ от $|A^2 - A_{cr}^2|$:

\begin{figure}[h]
	\includegraphics[scale=1.25]{img/result.jpg}
	\caption{$\eta$ от $|A^2 - A_{cr}^2|$}
\end{figure}

Из графика найдём, что $\eta \sim |A^2 - A_{cr}^2|^{0.6}$, то есть степень $b = 0.6 \pm 0.2$, что хорошо согласуется с предсказываемой теоретической зависимостью $b_{th} = 0.5$.


\subsection*{Вывод}
 
В данной работе была рассмотрена классическая теория фазовых переходов второго рода разработанная Л. Д. Ландау. Полученная физическая модель описывает широкий класс фазовых переходов:  парамагнетик-ферромагнетик или парамагнетик-антиферромагнетик, переходы в сегнетоэлектриках и некоторые другие.

Также была предложена наглядная механическая аналогия модели переходов второго рода. Результаты эксперимента согласуются с физической моделью в пределах погрешности.


\subsection*{Источники}

[1] \textit{Д.\,В. Сивухин.} Общий курс физики. Термодинамика и молекулярная физика.

\vspace{-.7\parskip}
[2] \textit{Л.\,Д. Ландау, Е.\, М. Лифшиц} Том VIII. Электродинамика сплошных сред.

\vspace{-.7\parskip}
[3] \textit{Л.\,Д. Ландау, Е.\, М. Лифшиц} Том V. Статистическая физика.

\end{document}